% Created by Jim Finnis
% Date Mon Aug 3 11:19:48 2020

\section{PCT data}
Document \verb+EXM-PC-DRW-ABU-0007_1.4_PCT_Drawing+ contains the following information,
all dimensions in mm:

\begin{itemize}
\item Large patches diameter: 29.70 (glass) 32.0 (outer)
\item Small patches diameter: 17.80 (glass) 20.0 (outer)
\item X from RH edge of small patches: 12.5, 33.5, 54.5 (separation = 21)
\item Y from T edge of small patches: 11, 32 (all mm)
\item X from RH edge of large patches: 17, 50
\item Y from T edge of large patches: 59
\end{itemize}


\section{Hough transform algorithm}
First algorithm is this, which works on a small section of the image due to the
slowness of the Hough transform:
\begin{itemize}
\item convert to grayscale (this is a bit weird, it's
$0.299r+0.587g+0.144b$ to match human perception for want of anything else; ideally
it would use something appropriate to the filter being used)
\item Canny edge detect with fairly arbitrary thresholds, generating boolean array
\item convert to [0,255] unsigned bytes
\item perform Hough ellipse detection (takes a while)
\item discard ellipses with very small major/minor axes (i.e. almost-lines)
\item discard ellipses with a high eccentricity ($(a/b)-1 \ge 0.2$, where $a$ and $b$ are
the axis lengths)
\item get the best 100 results
\end{itemize}
This sort of works, but breaks down when the edge detection produces a lot of small artifacts.
It's also excruciatingly slow.

\section{Blob detection algorithm}
OpenCV has a blob detector with a fair few tunable parameters. This produces pretty good results,
but may also produce a lot of false positives/negatives. Still, it's infinitely faster than the
Hough algorithm. \textbf{Parameters have to be picked quite carefully.} There are two small refinements:
\begin{itemize}
\item I cull ellipses which are more than 10 standard deviations away in either $x$ or $y$ from
the centroid of all ellipses.
\item Not being sure of maxthresh, I iterate through all the thresholds and keep the largest number of
matches, provided that number is $\le 8$ (count takes place after the cull above). This may not be ideal.
\end{itemize}


We now need to find the PCT within those ellipses. This is going to require fitting some kind of
model of the PCT to the data. The data consists of approximate centres and radii. 
Ideas:
\begin{itemize}
\item RANSAC? This is hard because the model is very unconstrained. Firstly, what \emph{is} the
model? I would argue a centre, a two unit vectors (fwd and up) normal (thus defining a plane) and a scale.
We can constrain some of these.
\item There should be two radii represented in the data: small and large. At most 6 ellipses are small, at most
2 are large. If there are significantly more or fewer different radii than that, perhaps we can tune the thresholds on the blob
detector and re-run; similarly if there are many more or fewer ellipses of the two different radii.
\end{itemize}
The \verb+getPerspectiveTransform+ method in OpenCV takes two sets of 
points: one is points the image, the other is points in the world. Both are 2D: the world points
are all assumed to be at $z=0$. 
